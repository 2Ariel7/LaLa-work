\documentclass[uplatex, dvipdfmx]{jsarticle}


\usepackage{tcolorbox}
\usepackage{color}
\usepackage{listings, plistings}

%% ノート/latexメモ
%% http://pepper.is.sci.toho-u.ac.jp/pepper/index.php?%A5%CE%A1%BC%A5%C8%2Flatex%A5%E1%A5%E2

% Java
\lstset{% 
  frame=single,
  backgroundcolor={\color[gray]{.9}},
  stringstyle={\sffamily \color[rgb]{0,0,1}},
  commentstyle={\itshape \color[cmyk]{1,0,1,0}},
  identifierstyle={\ttfamily}, 
  keywordstyle={\ttfamily \color[cmyk]{0,1,0,0}},
  basicstyle={\ttfamily},
  breaklines=true,
  xleftmargin=0zw,
  xrightmargin=0zw,
  framerule=.2pt,
  columns=[l]{fullflexible},
  numbers=left,
  stepnumber=1,
  numberstyle={\scriptsize},
  numbersep=1em,
  language={Java},
  lineskip=-0.5zw,
  morecomment={[s][{\color[cmyk]{1,0,0,0}}]{/**}{*/}},
  keepspaces=true,         % 空白の連続をそのままで
  showstringspaces=false,  % 空白字をOFF
}
%\usepackage[dvipdfmx]{graphicx}
\usepackage{url}
\usepackage[dvipdfmx]{hyperref}
\usepackage{amsmath, amssymb}
\usepackage{itembkbx}
\usepackage{eclbkbox}	% required for `\breakbox' (yatex added)
\usepackage{enumerate}
\usepackage[default]{cantarell}
\usepackage[T1]{fontenc}
\fboxrule=0.5pt
\parindent=1em

\makeatletter
\def\verbatim@font{\normalfont
\let\do\do@noligs
\verbatim@nolig@list}
\makeatother

\begin{document}

%\anaumeと入力すると穴埋め解答欄が作れるようにしてる。\anaumesmallで小さめの穴埋めになる。
\newcounter{mycounter} % カウンターを作る
\setcounter{mycounter}{0} % カウンターを初期化
\newcommand{\anaume}[1][]{\refstepcounter{mycounter}{#1}{\boxed{\phantom{aa}\textnormal{\themycounter}\phantom{aa}}}} %穴埋め問題の空欄作ってる。
\newcommand{\anaumesmall}[1][]{\refstepcounter{mycounter}{#1}{\boxed{\tiny{\phantom{a}\themycounter \phantom{a}}}}}%小さい版作ってる。色々改造できる。

%% 修正時刻: Sat Dec  4 22:03:26 2021


\section{こんな場合}

\begin{tcolorbox}
 [title=想定されるケース, fonttitle=\gtfamily]
 自宅でいろいろとコードを書いて試してみたいけれど、翌日は今日やったところから
 始めたいので、今日まで書いたコードは変更したくない。
\end{tcolorbox}

つまり、今日講習で書いたコードを使っていろいろ試してみたいが、コードを変更
したくない。となると、今日やったところまでのプロジェクト・フォルダを
別のワークスペースにコピーして、そこでやってみる。そういう方法が
考えられる。

もちろんそれでもいいのだけれど、Git の仕組みを使ってやってみることもできる。\\

\begin{tcolorbox}
 [title=Gitの状況, fonttitle=\gtfamily]
 Gitで管理しているフォルダ : ``LaLa-work''(ワークスペース) \\
 現在のブランチ : ``main''
\end{tcolorbox}

\section{計画}

以下のような計画でやってみる。

\begin{enumerate}
 \item 新しく ``rensyu'' というブランチを作成する。
 \item ``rensyu'' ブランチでいろいろと試してみる。
 \item ``rensyu'' ブランチに加えた変更を保存する。(Gitにアップする)
 \item 元の ``main'' ブランチに戻る。
\end{enumerate}


\section{新しく ``rensyu'' というブランチを作成する}

\subsection{現在のブランチ}

現在のブランチは以下のコマンドで確認できる。

\begin{tcolorbox}
 [title=現在のブランチを確認, fonttitle=\gtfamily]
$>$ git \ branch \\
\ $*$main
\end{tcolorbox}

現在は mainブランチのみである。

\subsection{ブランチを作成する}

ブランチ関係のコマンドは、とりあえずは以下。

\begin{tcolorbox}
 [title=ブランチの作成, fonttitle=\gtfamily]
$>$ git \ branch \ rensyu \quad (``rensyu'' というブランチを作成)
\end{tcolorbox}

\begin{tcolorbox}
 [title=現在のブランチを ``rensyu'' にする, fonttitle=\gtfamily]
$>$ git \ checkout \ rensyu
\end{tcolorbox}

\begin{tcolorbox}
 [title=ブランチを作成するとともに、そのブランチをカレントブランチとする,
 fonttitle=\gtfamily]
$>$ git \ checkout \ -b \ rensyu
\end{tcolorbox}

\begin{tcolorbox}
 [title=``rensyu''ブランチを削除する,
 fonttitle=\gtfamily]
$>$ git \ branch \ -d \ rensyu
\end{tcolorbox}

\begin{tcolorbox}
 [title=``rensyu''ブランチの名前を ``work''に変更する,
 fonttitle=\gtfamily]
$>$ git \ branch \ -m \ rensyu \ work
\end{tcolorbox}

3番目の''ブランチを作成するとともに、そのブランチをカレントブランチとする''が
手っとり早い。

\fbox{$>$ \ git \ checkout \ -b \ rensyu \ }

\vspace{3mm}
確認する

\vspace{3mm}
\begin{tabular}{|l|} \hline
$>$ git branch \\
\qquad main \\
\quad $*$ rensyu \\ \hline
\end{tabular}
\vspace{3mm}

\subsection{ブランチをプッシュする}

このブランチでいろいろ試したあと、リモート(Github)にアップしたいとする。

\fbox{\ git \ status \ } とすると、現在のブランチが表示されるので、
そのブランチを指定してプッシュすることになる。

(例) 現ブランチが ``rensyu'' である場合。

\vspace{3mm}
\begin{tabular}{|l|} \hline
$>$ \ git \ status \\
On branch rensyu \\ \hline
\end{tabular}
\vspace{3mm}

git status とすると、現在 rensyuブランチであることがわかる。

プッシュする場合の流れは以下である。

\vspace{3mm}
\begin{tabular}{|l|} \hline
$>$ git \ add \ .(ピリオド) \\
$>$ git \ commit \ -m \ ``Main.javaを追加'' \\
$>$ git \ push \ -u \ origin \ \underline{rensyu} \\ \hline
\end{tabular}
\vspace{3mm}

最後の push コマンドのところで ``rensyu''ブランチを指定している。

% \newpage
\vspace{10mm}
\section{学校でブランチを作成し、Github にプッシュした。それを自宅で
とりこみたい}

こういう状況を想定している。

\vspace{3mm}
学校 \quad
\begin{tabular}{|l|} \hline
$>$ git branch \\
\quad main \\
\quad rensyu \\ \hline
\end{tabular}
\vspace{3mm}

学校でブランチを作成したので、学校で git branch とすると、main と rensyu
の2つのブランチが表示される。


\vspace{3mm}
Github \quad
\begin{tabular}{|l|} \hline
main \\
rensyu \\ \hline
\end{tabular}
\vspace{3mm}

学校でブランチを作成したあと、Githubにプッシュしたので、Github には
main と rensyu が存在している。

自宅のパソコンで以下のコマンドを実行すると、以下のような表示になるはず
である。

\vspace{3mm}
\begin{tabular}{|l|} \hline
$>$ git branch -r \\
origin/HEAD -> origin/main \\
origin/main \\
origin/rensyu \\ \hline
\end{tabular}
\vspace{3mm}

この \fbox{\ git branch -r \ } というコマンドは、リモート(Github) に
存在するブランチを表示するコマンドである。
Githubには当然 main と rensyu の2つのブランチがある。

自宅のパソコンで以下のコマンドを実行すると、以下のような表示になる。

\vspace{3mm}
\begin{tabular}{|l|} \hline
$>$ git branch \\
$*$ main \\ \hline
\end{tabular}
\vspace{3mm}

まだ rensyu ブランチを取り込んでいないので、当然である。

\begin{tcolorbox}
[title=リモートの''rensyu''ブランチを取り込み、それをカレント・ブランチとする,
 fonttitle=\gtfamily]
$>$ git \ checkout \ -b \ rensyu \ origin/rensyu 
\end{tcolorbox}

こののち、以下のコマンドを実行して、確認する。

\vspace{3mm}
\begin{tabular}{|l|} \hline
$>$ git branch \\
\quad main \\
$*$ rensyu \\ \hline
\end{tabular}
\vspace{3mm}

% \newpage
\vspace{10mm}
\section{''rensyu''ブランチの変更を''main''に取り込む}

``rensyu''ブランチは試しにいろいろやってみるブランチなので、
それを ``main''ブランチに取り込むことはないかもしれないが、
そのやり方は以下である。



まず、現在のブランチを確認する。

\fbox{ \ git \ branch}

以下のように、``main''ブランチに戻ってから ``rensyu''ブランチの内容を取り込む

\begin{tcolorbox}
[title=リモートの''rensyu''ブランチを取り込み、それをカレント・ブランチとする,
 fonttitle=\gtfamily]
$>$ git \ checkout \ main \\
$>$ git \ merge \ rensyu \ -\--no-ff
\end{tcolorbox}

-\--no-ff オプションについては、以下を参照。

\begin{itemize}
 \item \href{https://qiita.com/nog/items/c79469afbf3e632f10a1}{gitのmerge --no-ff のススメ}
 \item \href{https://yu8mada.com/2018/08/15/the-difference-between-the-git-merge-options-ff-no-ff-and-ff-only/}{git merge オプションの --ff, --no-ff, --ff-only の違い}
\end{itemize}



\section{リモート(Github)に存在する''rensyu''ブランチを削除する}

``rensyu''ブランチもその役目を果たし、削除する日がやってきた。

この場合、ローカルの''rensyu''ブランチを削除するとともに、リモートの
''rensyu''ブランチも削除しなくてはならない。

手順は以下である。

\begin{tcolorbox}
[title=リモート(Github)に存在する''rensyu''ブランチを削除する,
 fonttitle=\gtfamily]
$>$ git \ branch \ -d \ rensyu \qquad (ローカルの''rensyu''ブランチを削除) \\
$>$ git \ branch \ -r  \qquad (リモートのブランチを確認) \\
 origin/HEAD -> origin/main \\
 origin/main \\
 origin/rensyu \\
$>$ git \ push \ -\--delete \ origin \ rensyu  \qquad (リモートのブランチを削除) 
\end{tcolorbox}





\end{document}

%% 修正時刻: Sat May  2 15:10:04 2020


%% 修正時刻: Mon Dec  6 07:40:31 2021
