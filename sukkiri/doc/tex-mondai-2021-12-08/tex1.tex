\documentclass[dvipdfmx]{jsarticle}


\usepackage{tcolorbox}
\usepackage{color}
\usepackage{listings, plistings}

%% ノート/latexメモ
%% http://pepper.is.sci.toho-u.ac.jp/pepper/index.php?%A5%CE%A1%BC%A5%C8%2Flatex%A5%E1%A5%E2

% Java
\lstset{% 
  frame=single,
  backgroundcolor={\color[gray]{.9}},
  stringstyle={\sffamily \color[rgb]{0,0,1}},
  commentstyle={\itshape \color[cmyk]{1,0,1,0}},
  identifierstyle={\ttfamily}, 
  keywordstyle={\ttfamily \color[cmyk]{0,1,0,0}},
  basicstyle={\ttfamily},
  breaklines=true,
  xleftmargin=0zw,
  xrightmargin=0zw,
  framerule=.2pt,
  columns=[l]{fullflexible},
  numbers=left,
  stepnumber=1,
  numberstyle={\scriptsize},
  numbersep=1em,
  language={Java},
  lineskip=-0.5zw,
  morecomment={[s][{\color[cmyk]{1,0,0,0}}]{/**}{*/}},
  keepspaces=true,         % 空白の連続をそのままで
  showstringspaces=false,  % 空白字をOFF
}
%\usepackage[dvipdfmx]{graphicx}
\usepackage{url}
\usepackage[dvipdfmx]{hyperref}
\usepackage{amsmath, amssymb}
\usepackage{itembkbx}
\usepackage{eclbkbox}	% required for `\breakbox' (yatex added)
\usepackage{enumerate}
\usepackage[default]{cantarell}
\usepackage[T1]{fontenc}
\fboxrule=0.5pt
\parindent=1em

\makeatletter
\def\verbatim@font{\normalfont
\let\do\do@noligs
\verbatim@nolig@list}
\makeatother

\begin{document}

%\anaumeと入力すると穴埋め解答欄が作れるようにしてる。\anaumesmallで小さめの穴埋めになる。
\newcounter{mycounter} % カウンターを作る
\setcounter{mycounter}{0} % カウンターを初期化
\newcommand{\anaume}[1][]{\refstepcounter{mycounter}{#1}{\boxed{\phantom{aa}\textnormal{\themycounter}\phantom{aa}}}} %穴埋め問題の空欄作ってる。
\newcommand{\anaumesmall}[1][]{\refstepcounter{mycounter}{#1}{\boxed{\tiny{\phantom{a}\themycounter \phantom{a}}}}}%小さい版作ってる。色々改造できる。

%% 修正時刻: Sat Dec  4 22:03:26 2021


\section{問題}

\subsection{p.392のコードを入力する}

\begin{lstlisting}[caption=Item.java]
 package world;

 public class Item {
   public String name;
   public int price;

   public Item(String name) {
     this.name = name;
     this.price = 0;
   }

   public Item(String name, int price) {
     this.name = name;
     this.price = price;
   }
 }
\end{lstlisting}

Weapon.javaは以下でよい。

\begin{lstlisting}[caption=Weapon.java]
 package world;
 
 public class Weapon extends Item {

 }
\end{lstlisting}

p.393のコードを以下のように入力する。

\begin{lstlisting}[caption=Main02.java]
 package chap10;

 import world.Weapon;

 public class Main02 {

   public static void main(String[] args) {
     Weapon w = new Weapon();
     System.out.println(w.name + ":" + w.price);
   }
 }
\end{lstlisting}

\subsection{表示されるエラーの意味を考える}

\begin{quote}
「暗黙的スーパー・コンストラクター Item() は、デフォルト・コンストラクターに
ついては未定義です。明示的コンストラクターを定義する必要があります」
\end{quote}

つまり、Weapon.java は内部に何の記述も無いが、
Main02.java で new Weapon() が実行されているので、Weapon.javaの中の
引数なしのコンストラクター(暗黙的)が実行される。

Weapon.javaは子クラスなので、親クラスの引数なしコンストラクターが実行される。
したがって、Weapon.javaは実質的に以下のコードになる。

\begin{lstlisting}[caption=Weapon.java]
 public class Weapon extends Item {
   public Weapon() {
     super();
   }
 }
\end{lstlisting}

ところが、Item.java には、引数ありのコンストラクターがあるため、
引数なしのコンストラクターは定義されないと「無い」ものとして扱われる。
だから、エラーが出る。

\subsection{エラーを解決する}

エラーを解決するには、以下の3通りが考えられる。それぞれ実行してみて、確認してみる。

\begin{enumerate}
 \item Item.java に 引数なしのコンストラクターを記述する。
 \item Weapon.java の コンストラクターを修正する。p.394の方法がこれである。
 \item Main02.java を new Weapon("バズーカ") とする。これは Weapon.java の
コンストラクターも修正する必要がある。
\end{enumerate}


\section{問題}

Item.java に以下のメソッドをつくる。

\begin{lstlisting}[caption=Item.java]
 public void takeItem() {
   System.out.println(this.name + "をとる。");
 }
 public void useItem() {
   System.out.println(this.name + "を使う。");
 }
\end{lstlisting}

Weapon.java に以下のメソッドをつくる。

\begin{lstlisting}[caption=Weapon.java]
 public void useItem() {
   System.out.println(this.name + "をぶっ放す。");
 }
\end{lstlisting}

これを Main02.java で使ってみる。


\section{問題}

\textsf{chap09} に出現しているエラーを解決する。
エラーのほとんどは Hero.java や Matango.java のメソッドに関するものだから、
Hero.java や Matango.java に不足しているメソッドを追加してやればよい。


\end{document}

%% 修正時刻: Sat May  2 15:10:04 2020


%% 修正時刻: Sat Dec  4 21:47:56 2021
